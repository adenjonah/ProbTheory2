\textbf{Problem 2: Independence}

To check if $X$ and $Y$ are independent, we need to verify whether $P(X=i, Y=j) = P(X=i) \cdot P(Y=j)$ for all $i,j$.

First, calculate the marginal pmfs:

\textbf{Marginal pmf of $X$:}
\begin{align}
P(X=1) &= P(X=1,Y=1) + P(X=1,Y=2) + P(X=1,Y=3) = \frac{1}{18} + \frac{1}{9} + \frac{1}{6} = \frac{1+2+3}{18} = \frac{1}{3}\\
P(X=2) &= P(X=2,Y=1) + P(X=2,Y=2) + P(X=2,Y=3) = \frac{1}{9} + \frac{1}{6} + \frac{1}{18} = \frac{2+3+1}{18} = \frac{1}{3}\\
P(X=3) &= P(X=3,Y=1) + P(X=3,Y=2) + P(X=3,Y=3) = \frac{1}{6} + \frac{1}{18} + \frac{1}{9} = \frac{3+1+2}{18} = \frac{1}{3}
\end{align}

\textbf{Marginal pmf of $Y$:}
\begin{align}
P(Y=1) &= P(X=1,Y=1) + P(X=2,Y=1) + P(X=3,Y=1) = \frac{1}{18} + \frac{1}{9} + \frac{1}{6} = \frac{1+2+3}{18} = \frac{1}{3}\\
P(Y=2) &= P(X=1,Y=2) + P(X=2,Y=2) + P(X=3,Y=2) = \frac{1}{9} + \frac{1}{6} + \frac{1}{18} = \frac{2+3+1}{18} = \frac{1}{3}\\
P(Y=3) &= P(X=1,Y=3) + P(X=2,Y=3) + P(X=3,Y=3) = \frac{1}{6} + \frac{1}{18} + \frac{1}{9} = \frac{3+1+2}{18} = \frac{1}{3}
\end{align}

\textbf{Independence check:}
If independent, then $P(X=i, Y=j) = P(X=i) \cdot P(Y=j) = \frac{1}{3} \cdot \frac{1}{3} = \frac{1}{9}$ for all $i,j$.

However, from the table:
- $P(X=1, Y=1) = \frac{1}{18} \neq \frac{1}{9}$
- $P(X=1, Y=3) = \frac{1}{6} \neq \frac{1}{9}$
- $P(X=2, Y=2) = \frac{1}{6} \neq \frac{1}{9}$
- And several other entries differ from $\frac{1}{9}$

Since $P(X=i, Y=j) \neq P(X=i) \cdot P(Y=j)$ for multiple pairs $(i,j)$, we conclude that $X$ and $Y$ are \textbf{NOT independent}.

\textbf{Verification:} All marginal probabilities sum to 1 ✓
