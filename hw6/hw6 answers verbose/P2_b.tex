% Problem 2(b): Posterior after n rolls of 7

\textbf{Posterior probability for each die after $n$ rolls of 7:}

Only the 8, 12, and 20-sided dice can roll 7. For a $k$-sided die with $k \geq 7$:
$$P(\text{$n$ rolls of 7} \mid k\text{-sided}) = \left(\frac{1}{k}\right)^n$$

Let $S_n = \left(\frac{1}{8}\right)^n + \left(\frac{1}{12}\right)^n + \left(\frac{1}{20}\right)^n$ be the normalizing constant.

\textbf{Posterior formulas:}
\begin{align*}
P(\text{4-sided} \mid \text{data}) &= 0 \\
P(\text{6-sided} \mid \text{data}) &= 0 \\
P(\text{8-sided} \mid \text{data}) &= \frac{(1/8)^n}{S_n} \\
P(\text{12-sided} \mid \text{data}) &= \frac{(1/12)^n}{S_n} \\
P(\text{20-sided} \mid \text{data}) &= \frac{(1/20)^n}{S_n}
\end{align*}

\textbf{Limits as $n \to \infty$:}

Multiplying numerator and denominator by $8^n$:
$$P(\text{8-sided} \mid \text{data}) = \frac{1}{1 + (8/12)^n + (8/20)^n} = \frac{1}{1 + (2/3)^n + (2/5)^n}$$

As $n \to \infty$: $(2/3)^n \to 0$ and $(2/5)^n \to 0$, so:
$$\boxed{P(\text{8-sided} \mid \text{data}) \to 1, \quad P(\text{12-sided} \mid \text{data}) \to 0, \quad P(\text{20-sided} \mid \text{data}) \to 0}$$

\textbf{Explanation:} The 8-sided die has the highest probability of rolling 7 ($1/8 > 1/12 > 1/20$). Observing many 7s is most consistent with the 8-sided die, so its posterior probability converges to 1.

\textbf{Verification:} For $n=100$, numerically $P(\text{8-sided}) \approx 1.0$, $P(\text{12-sided}) \approx 2.5 \times 10^{-18}$, $P(\text{20-sided}) \approx 1.6 \times 10^{-40}$. $\checkmark$

